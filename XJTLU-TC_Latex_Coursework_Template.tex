% A template for XJTLU CourseWork by LaTeX
% Author: Taimingwang Liu

%%%%%%%% Parameters to Change

% Basic information
\newcommand{\moduleName}{XXX000TC MODULE\_NAMEs}
\newcommand{\authorName}{AUTHOR\_NAME}
\newcommand{\authorID}{AUTHOR\_ID}
\newcommand{\reportDate}{\today}
% \newcommand{\reportDate}{November 13, 2023}

% The vertical offset of the title
\newcommand{\titleVerticalOffset}{-1.5cm}

% Picture Paths
\newcommand{\XJTLU}{pics/logo/xjtlu_logo_blue.pdf} % XJTLU logo
\newcommand{\coverPic}{pics/avator.png} % Cover picture(Optional)

% Header and Footer
% Width
\newcommand{\headerWidth}{2pt} % Width of the header line
\newcommand{\footerWidth}{2pt} % Width of the footer line
% Contents
\newcommand{\headL}{\leftmark} % Left header
\newcommand{\headC}{} % Center header
\newcommand{\headR}{\moduleName} % Right header
\newcommand{\footL}{} % Left footer
\newcommand{\footC}{\thepage} % Center footer
\newcommand{\footR}{\authorName \authorID} % Right footer

%%%%%%%% Boolean Parameters
% Whether to use MS Word margin
\newif\ifisMSWordMargin
\isMSWordMargintrue % true
% \isMSWordMarginfalse % false

% Whether to use header and footer
\newif\ifisHeaderFooter
\isHeaderFootertrue % true
% \isHeaderFooterfalse % false


%%%%%%%%%%%%%%%%
%    Others    %
%%%%%%%%%%%%%%%%
% 1. You can customize color in "settings.sty", which is the style I 

%%%%%%%% %%%%%%%% %%%%%%%% %%%%%%%% %%%%%%%% %%%%%%%% %%%%%%%% %%%%%%%% 
\documentclass{article}

\usepackage{settings}

% Indent first paragraph
\usepackage{indentfirst}
\setlength{\parindent}{2em}

% Get larger line spacing in table
\newcommand{\tablespace}{\\[1.25mm]}
\newcommand\Tstrut{\rule{0pt}{2.6ex}}         % = `top' strut
\newcommand\tstrut{\rule{0pt}{2.0ex}}         % = `top' strut
\newcommand\Bstrut{\rule[-0.9ex]{0pt}{0pt}}   % = `bottom' strut


%%%%%%%%%%%%%%%%%
%     Title     %
%%%%%%%%%%%%%%%%%

\title{
    \includegraphics[width=0.9\textwidth]{\XJTLU}\\
    \vspace{1cm}
    \moduleName
    }

\author{
    \authorName \\
    \authorID
    }

\date{\reportDate}

\begin{document}
\maketitle

% Cover picture
\vspace{2cm}
\picHereSimple{\coverPic}{0.7\textwidth}

\newpage
%%%%%%%%%%%%%%%%%%%%%%%%%
%   Table of Contents   %
%%%%%%%%%%%%%%%%%%%%%%%%%
\tableofcontents

\newpage
%%%%%%%%%%%%%%%%%%%%
%   Introduction   %
%%%%%%%%%%%%%%%%%%%%
\section{Introduction}
\subsection{Background}
Greetings! This is a template for XJTLU coursework by \LaTeX. \cite{example1}
I make this template because of \href{https://github.com/feimax/latex_template_for_xjtlu_eee_light}{\underline{the excellent work} of \textit{Dr.Cheng} and the EEE students of \textit{XJTLU}}. 
\textit{Dr.Cheng} has received his PhD degree in Liverpool University and now works in \textit{XJTLU}. 
The repository above was also used for his doctoral thesis.\cite{example2}\par
Kind as him, \href{https://www.xjtlu.edu.cn/en/study/departments/school-of-advanced-technology/communications-and-networking/department-staff/academic-staff/staff/fei-cheng}{\textit{\underline{Dr.Cheng}}} shared his template on \textit{Github} for every \textit{XJTLUer}. 
Meanwhile, he also worked on a tutorial about how to use \LaTeX. (It's a pity that \href{http://blog.feieee.com/latex}{\underline{his blog}} is now closed.)
Luckily, I found this burried treasure today(November 13, 2023).\par

\subsection{Why I make this template}
Inspired by his \textit{Open Source Spirit}, and considering his work is too sophisticated, I decided to make \textbf{a simpler template} for XJTLU coursework, 
which is not so official but still useful.

\newpage
%%%%%%%%%%%%%%%%%%%%
%      Part 1      %
%%%%%%%%%%%%%%%%%%%%
\section{What's in this template}
\subsection{Pre-defined Commands}
\noindent This template provides a command \textit{\textbf{picHere}} to insert pictures, with 4 parameters:
\begin{enumerate}
    \item ``\textit{path}'': the path of the picture
    \item ``\textit{width}'': the width of the picture(percentage of the text width is recommended)
    \item ``\textit{caption}'': the caption of the picture
    \item ``\textit{label}'': the label of the picture(used for reference)
\end{enumerate}

\noindent For example, we can use ``\textit{\textbf{\textbackslash picHere\{pics/TC.png\}\{0.5\textbackslash textwidth\}\{TC Campus\}\{TC\}}}'' to insert a picture like this:
\picHere{pics/TC.png}{0.8\textwidth}{TC Campus}{TC}

This template also provides a command \textit{\textbf{picHereSimple}} to insert pictures, 
with only the first 2 parameters. 
This one is used for pictures without caption and label. (e.g . the cover picture)

\subsection{Global Settings}
At the beginning of this file, there are some parameters you can change to customize your report. 
Including ``Basic Information''(e.g. \textit{moduleName}, \textit{authorName}...), 
``The vertical offset of the title''(e.g. \textit{titleVerticalOffset}), ``Picture Paths''(e.g. \textit{XJTLU}, \textit{coverPic}),
``Header and Footer''(e.g. \textit{headerWidth}, \textit{footerWidth}...).\par
Moreover, the contents of \textit{Left}, \textit{Center} and \textit{Right} in the header and footer can also be changed easily by modifying the corresponding parameters.

\subsection{Boolean Parameters}
This is my favorite part. This template provides some boolean parameters to customize the style of the report.
Currently, you can choose whether to use \textit{MS Word Margin} and \textit{Header and Footer}.\par
For example, if you want a report with \textit{MS Word Margin} and \textit{Header and Footer}, 
just keep the ``\textit{\textbf{\textbackslash isMSWordMargintrue}}'' and ``\textit{\textbf{\textbackslash isHeaderFootertrue}}''.(They are default settings)\par
If you want to disable any of them, just change the postfix ``\textit{true}'' to ``\textit{false}'', \\ 
e.g. ``\textit{\textbf{\textbackslash isMSWordMargintrue}}'' \(\rightarrow\) ``\textit{\textbf{\textbackslash isMSWordMarginfalse}}''.

\subsection{Code}
\noindent A color palette for code is defined in \textit{settings.sty}, which is the style I used in my \textit{VS Code}.
\subsubsection{\textit{hello.cpp}}
\lstinputlisting[caption=hello.cpp]{./code/hello.cpp}

\subsubsection{\textit{multiplication.py}}
\lstinputlisting[caption=multiplication.py]{./code/multiplication.py}

\noindent Note: A parameter ``\textit{language}'' in the command ``\textit{\textbf{\textbackslash lstset}}'' is used to specify the language of the code. 
\textbf{Please change it to the language you use.} 
Wrong type of language may cause some error of the the key word color.

\newpage
%%%%%%%%%%%%%%%%%%%%
%     Appendix     %
%%%%%%%%%%%%%%%%%%%%
\section{Appendix}
\noindent Pictures used in this report:
\begin{enumerate}
    \item The \textit{XJTLU} Logo is from \href{https://github.com/feimax/latex_template_for_xjtlu_eee_light}{\underline{the \textit{Github} repository of \textit{Dr.Cheng}}}
    \item The cover picture is the avator of author, which is made by \textit{Stable Diffusion}. (I call her \textit{Chicarlet}.)
    \item The picture of \textit{TC Campus} is from \href{https://www.xjtlu.edu.cn/en/study/departments/entrepreneurship-and-enterprise-hub}{\underline{the official website of \textit{XJTLU}}}
\end{enumerate}

These pictures are used for \textbf{academic purpose only}. Many thanks to the authors.

\newpage
%%%%%%%%%%%%%%%%%%%%
%     Reference    %
%%%%%%%%%%%%%%%%%%%%
% \bibliographystyle{apalike} % APA: apacite | IEEE: ieeetr | Plain: plain
\bibliographystyle{ieeetr}
\bibliography{reference}

\end{document}
